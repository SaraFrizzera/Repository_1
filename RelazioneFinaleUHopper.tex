\documentclass[a4paper,11pt]{article} 

\usepackage[english]{babel}
\usepackage[utf8]{inputenc}
\usepackage[T1]{fontenc}
\usepackage{graphicx}
\usepackage{wrapfig}
\usepackage{amsmath}
\usepackage{enumerate}
\usepackage{subfigure}
\usepackage{fancyhdr}
\usepackage{amsthm}
\usepackage{amsfonts}
\usepackage[utf8]{inputenc}
\usepackage[T1]{fontenc}
\usepackage{calligra}
\usepackage{makeidx}
\usepackage{amssymb}
\usepackage{makecell}
\usepackage{amscd} %x diagrammi matematici
\usepackage{marvosym}
\usepackage{enumitem}
\theoremstyle{plain}
\usepackage[a4paper, inner=1.5cm, outer=2.5cm, top=2.5cm, 
bottom=2.5cm, bindingoffset=1cm]{geometry} 
\usepackage{algorithm}
\usepackage{algpseudocode}
\usepackage{pifont}
\newcommand{\sid}[1]{\color{red} #1 \color{black} }
\newcommand{\barbi}[1]{\color{blue} #1 \color{black} }
\newcommand{\simo}[1]{\color{violet} #1 \color{black}}
\usepackage{mathtools}
\newcommand{\fsl}[1]{\ensuremath{\mathrlap{\!\not{\phantom{#1}}}#1}}% \fsl{<symbol>}

\begin{document}
\title{\huge RELAZIONE FINALE TIROCINIO}
\author{Sara Frizzera}
\date{15/09/2018}
\maketitle

\section*{Dati}
SARA FRIZZERA - [MAT. 185274]\\
Dipartimento di Ingegneria e Scienza dell’Informazione\\
Corso di studio:	[0330G] - Ingegneria dell'informazione e Organizzazione D'impresa\\
Ordinamento: [0330GR08] - Ordinamento 2008\\
Email: sara.frizzera@studenti.unitn.it\\
Recapito telefonico: +39 388 9512366\\
Progetto di tirocinio: “A chatbot for the Internet of Things” aka "Talk2Fiesta"

\section*{U-Hopper}
U-Hopper è un'azienda R\&D che si occupa di produzione software, consulenza informatica e attività connesse.
Utilizza le competenze maturate dal team nell’ambito dei big data analytics, delle tecnologie semantiche, delle tecnologie mobili e della localizzazione indoor.
E' nata a Trento a fine 2010 ed è diventata operativa nel corso del 2012. Nell’anno 2013 è entrata nella business community dell’EIT ICT Labs che, attraverso un team europeo di business developers, supporta l'accelerazione delle start-up innovative sul mercato europeo.
Tra i propri clienti U-Hopper annovera IGP Decaux, Telecom Italia, Groupalia, Trenta e Trentino Networks. U-Hopper è attiva nello sviluppo di tecnologie e prodotti innovativi in partnership con IBM, SINTEF, Quuppa, EVRYTHNG, Snowflake Software, Trilogis ed altri leader del settore.

\section*{Progetto “A chatbot for the Internet of Things”}

Il progetto Talk2Fiesta nasce come servizio di informazione conversazionale per Fiesta-IoT.

\subsection*{Fiesta-IoT}
Fiesta IoT è un’entità finanziata dall’Unione Europea, il cui obiettivo è quello di condividere e riutilizzare i dati provenienti da testbeds IoT eterogenei, basandosi su soluzioni semantiche. \\ 
La piattaforma di Fiesta garantisce l’accesso a 10 testbeds, che forniscono dati tramite sensori posizionati all'interno e all'esterno degli edifici in diverse città europee.\\
Il progetto Talk2Fiesta opera con due di questi testbeds:
\begin{enumerate}
\item SmartSantander: fornisce accesso ai sensori di temperatura, umidità, illuminazione, rumore ecc. nella città di Santander, in Spagna.
\item SmartICS: fornisce accesso ai sensori di temperatura, umidità, illuminazione e presenza umana nell’università di Surrey, in Inghilterra.
\end{enumerate} 
Lo scopo che si pone Fiesta è quello di trovare soluzioni all'eterogeneità dei BigData. La diversificazione dei dati IoT – infatti - rende difficile l'interoperabilità di settori verticali differenti. Ciò comporta la formazione di “data silos”, un decremento dell'efficienza e del potenziale effettivo di questi dati.
La soluzione proposta da Fiesta si basa sull'utilizzo della tassonomia e delle ontologie per classificare i meta-dati e garantire l'interoperabilità semantica.


\subsection*{Attività svolte}
Lo scopo del progetto Talk2Fiesta è la creazione di un chat-bot per l’estrazione di dati e conoscenze utilizzate per rispondere alle domande degli utenti, tramite richieste alle APIs di Fiesta-IoT. Al contempo esegue test di convalida con SmartSantander e SmartICS, sperimentando le scoperte semantiche, la disponibilità, affidabilità e latenza del servizio fornito da Fiesta-IoT. \\
Per l'implmentazione del chatbot è stato utilizzato il framework Wit.ai.\\
Wit è basato sul Natural Language Processing (NLP) ed è in grado di trasformare le frasi in dati strutturati.
L'allenamento del chat-bot è mirato all'individuazione di entità specifiche - data/ora, città e aggregazione - e di intenti - temperatura, umidità, illuminazione, rumore. \\
Ogni intento viene poi collegato ad un'azione, implementata in Python, che gestisce le entità e manda una richiesta specifica alle APIs di Fiesta. \\
Il codice che gestisce la logica del chatbot è stato implementato interamente in Python e integrato con le librerie di Flask, Json e Thread.\\
Le richieste alle APIs di Fiesta sono state implementate in SparQL.\\
Segue uno schema che rappresenta la struttura del progetto:
\vspace{0.2cm}
\begin{figure}[!h]
\centering
\includegraphics[scale=0.9]{scheme.png}
\caption{Struttura del progetto}
\end{figure}
\vspace{1cm}
\begin{enumerate}
\item L'utente invia un messaggio tramite un framework di messaggistica all'API del server U-Hopper. 
\item U-Hopper invia il testo a Wit.ai per il riconoscimento di entità e intenti.
\item Dopo aver ricevuto le informazioni aggiuntive da Wit.ai, è possibile associare un intento ad un'azione e quindi inviare un richiesta SparQL a Fiesta.
\item Con i dati forniti da Fiesta è possibile costruire una risposta alla domanda dell'utente e inviarla al framework di messaggistica
\end{enumerate}


\section*{Raggiungimento obbiettivi}
Gli obbiettivi raggiunti durante il tirocinio sono:
\begin{enumerate}
\item Acquisizione competenze relative a framework per l'implementazione e il deployment di chatbot 
\item Acquisizione competenze nella programmazione ad oggetti in Python
\item Acquisizione competenze nella creazione e gestione di server in Flask
\item Acquisizione conoscenze relative ai protocolli per l'interfacciamento con dispositivi IoT 
\item Acquisizione competenze di comunicazione interpersonale e metodo di lavoro in un ambiente internazionale e multiculturale 
\item Acquisizione competenze nella gestione, suddivisione e condivisione del lavoro (SourceTree)
\item Acquisizione conoscenza di metodologie di lavoro agile 
\item Acquisizione competenze di lavoro in team 
\end{enumerate}
\section*{Considerazioni finali}
Il percorso di tirocinio è stato stimolante sia dal punto di vista formativo che da quello personale.
Gli obbiettivi prefissati sono stati largamente soddisfatti.\\
Il giudizio finale è sicuramente positivo.



\end{document}